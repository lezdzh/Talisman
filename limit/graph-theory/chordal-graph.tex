\paragraph{定义} 我们称连接环中不相邻的两个点的边为弦. 一个无向图称为弦图, 当
图中任意长度都大于3的环都至少有一个弦. 弦图的每一个诱导子图一定是
弦图.

\paragraph{单纯点} 设$N(v)$表示与点$v$相邻的点集. 一个点称为单纯点当$v+N(v)$的
诱导子图为一个团. 引理: 任何一个弦图都至少有一个单纯点, 不是完全图
的弦图至少有两个不相邻的单纯点.

\paragraph{完美消除序列}	一个序列$v_1, v_2, \dots, v_n$满足$v_i$在$v_i, v_{i+1}, \dots, v_n$的诱导子
图中为一个单纯点. 一个无向图是弦图当且仅当它有一个完美消除序列.

\paragraph{最大势算法} 最大势算法能判断一个图是否是弦图. 从$n$到1的顺序依次给
点标号 标号为$i$的点出现在完美消除序列的第$i$个 . 设$label_i$表示第i个点
与多少个已标号的点相邻, 每次选择$label_i$最大的未标号的点进行标号.
然后判断这个序列是否为完美序列. 如果依次判断$v_i$在$v_{i+1}, \dots, v_n$中
所有与vi相邻的点是否构成一个团, 时间复杂度为$O(nm)$. 考虑优化,
设$v_{i+1}, \dots, v_n$中所有与$v_i$相邻的点依次为$v_{j1},\dots,v_{jk}$. 只需判断$v_{j1}$是否
与$v_{j2},\dots,v_{jk}$相邻即可. 时间复杂度$O(n + m)$.

\paragraph{弦图的染色} 按照完美消除序列中的点倒着给图中的点贪心染尽可能最小
的颜色, 这样一定能用最少的颜色数给图中所有点染色. 弦图的团数=染色
数.

\paragraph{最大独立集} 完美消除序列从前往后能选就选. 最大独立集=最小团覆盖.

\lstinputlisting{graph-theory/chordal-graph.cpp}