\paragraph{有源汇上下界费用流}
转换为求无源汇上下界最小费用可行循环流, 通过$T \rightarrow S$连边, 流量上下界为(原总流量, $\infty$)。

\paragraph{无源汇上下界最小费用可行循环流}
    在原基础上再新增一个超级源点 $supS, supT$, 构造只有上界的网络。

    对于原图的每一条边 $(u, v)$ , 再新图中添加一条 $supS\rightarrow v$ 流量为 $u, v$ 流量下界的边, 一条 $u\rightarrow supT$ 流量为 $u, v$ 流量下界的边, 一条 $u\rightarrow v$流量为 $u, v $流量上界-流量下界的边。

    做从$ supS\rightarrow supT$ 的最小费用流, 限定到达 $supT$ 的流量为满流(即 $supS$ 所有出边的流量和)。此即为答案。
    
    HINT: 原图中所有未提及的边费用都应记为 0 。新图中的重新构造的边的费用等同原图中对应边的费用。

\paragraph{上下界网络流}
$B(u, v) $表示边 $(u, v)$ 流量的下界,$C(u, v)$ 表示边 $(u, v) $流量的上界,$F (u, v)$ 表示边$ (u, v)$
的流量。
设 $G(u, v) = F (u, v) − B(u, v)$,显然有$0 \leq G(u, v) \leq C(u, v) − B(u, v)$

\subparagraph{无源汇的上下界可行流}
建立超级源点 $S_∗$ 和超级汇点 $T_∗$ ,对于原图每条边$ (u, v) $在新网络中连如下三条边: 
$S_∗ \rightarrow  v$, 容量为 $B(u, v)$;\\
 $u \rightarrow  T_∗$ ,容量为 $B(u, v)$;\\
 $u \rightarrow  v$,容量为 $C(u, v) − B(u, v)$。\\
最后求新网络的最大流,判断从超级源点 $S_∗$ 出发的边是否都满流即可,边$ (u, v) $的最终解中的实际流量为
$G(u, v) + B(u, v)$。

\subparagraph{有源汇的上下界可行流}
从汇点$ T $到源点$ S $连一条上界为 $\infty$,下界为 0 的边。按照无源汇的上下界可行流一样做
即可,流量即为$T \rightarrow S $边上的流量。

\subparagraph{有源汇的上下界最大流}
\begin{enumerate}
\item 在有源汇的上下界可行流中,从汇点$ T $到源点$ S $的边改为连一条上界为 $\infty$,下界为 $x$ 的
边。$x$ 满足二分性质,找到最大的$ x $使得新网络存在无源汇的上下界可行流即为原图的最大流。
\item 从汇点$ T $到源点$ S $连一条上界为 $\infty$,下界为 0 的边,变成无源汇的网络。按照无源汇的上下界可行流的方法,建立超级源点 $S_∗$ 和超级汇点 $T_∗$ ,求一遍 $S_∗ \rightarrow  T_∗$ 的最大流,再将从汇点$ T $到源点$ S $的这条边拆掉,求一次$S \rightarrow  T$的最大流即可。
\end{enumerate}

\subparagraph{有源汇的上下界最小流}
\begin{enumerate}
\item 在有源汇的上下界可行流中,从汇点$ T $到源点$ S $的边改为连一条上界为 $x$,下界为 0 的边。$x$ 满足二分性质,找到最小的 $x$ 使得新网络存在无源汇的上下界可行流即为原图的最小流。
\item 按照无源汇的上下界可行流的方法,建立超级源点 $S_∗$ 与超级汇点 $T_∗$ ,求一遍 $S_∗ \rightarrow T_∗$ 的最大流,但是注意这一次不加上汇点$ T $到源点$ S $的这条边,即不使之改为无源汇的网络去求解。求完后,再加上那条汇点$ T $到源点$ S $上界 $\infty$ 的边。因为这条边下界为 0,所以$S_∗$ ,$T_∗$ 无影响,再直接求一次 $S_∗ \rightarrow T_∗$ 的最大流。若超级源点 $S_∗$ 出发的边全部满流,则$T \rightarrow S $边上的流量即为原图的最小流,否则无解。
\end{enumerate}
